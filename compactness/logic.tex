
\documentclass{jarticle}
\begin{document}


命題論理のコンパクト性定理 \\

$ 論理式の集合 \Gamma が有限充足可能 \Rightarrow \Gamma が充足可能 $

を証明します

有限充足可能であるとは、$ \Gamma $ の任意の有限部分集合が充足可能であることと定義します

$ また、集合  \Gamma   が充足可能であるとは、任意の論理式 \phi \in \Gamma に対して V(\phi) = 1$

とするような真理値割り当て関数$ V $が存在することと定義します.

証明の方針としては、 $ \Gamma  を部分集合とする \Delta \supset \Gamma $ を考えるとそのような関数 $ V $ が存在することがわかる

という方法で行います. \\


まず命題論理の論理式とその解釈(意味論)について定義します

命題論理$L$は以下の記号からなる

論理結合子: $ \lnot , \land $
命題変数: $ p_0 , p_1 , ... $
補助記号: $ ( , ) $


論理式の構成
1. 命題変数 $p_0 , p_1 , ...$ は論理式である
2. $\phi を論理式とするとき \lnot \phi は論理式である$
3. $\phi , \psi を論理式とするとき (\phi \land \psi) は論理式である$


論理式の真偽

写像 $ \sigma : \{p_0 , p_1 , ...\} \rightarrow \{0,1\} $ を解釈とする

$ \sigma $をある解釈として論理式の真偽は次の写像により定まる

$ V_{\sigma} : P \rightarrow \{0,1\}  \\
V_{\sigma}(p_i) = \sigma(p_i) \\
V_{\sigma}(\lnot \phi) = 1 - V_{\sigma}(\phi) \\
V_{\sigma}(\phi \land \psi) は V_{\sigma}(\psi) = V_{\sigma}(\phi) = 1 の時 1 それ以外は 0
$


では定理

$ 論理式の集合 \Gamma が有限充足可能 \Rightarrow \Gamma が充足可能 $

を証明します


まず集合の列  $ \Delta_0 , \Delta_1 , ... $ を定義します \\


$\Delta_0 = \Gamma$ \\
$\Delta_n \cup  \{\phi_{n+1}\} が有限充足可能なら \Delta_{n+1} = \Delta_n \cup \ \{\phi_{n+1}\}$ \\
$\Delta_n \cup \{\phi_{n+1}\} が有限充足可能でないなら \Delta_{n+1} = \Delta_n \cup \ \{\lnot \phi_{n+1}\}$ \\

上の論理式 $ \phi_i $ についてですが、 整論理式が列挙可能で一列に並べることが出来るという事を使っています

その論理式を順にとってきて、1つ前の集合に加えて次の集合を作るという操作を行なっていきます





\end{document}




